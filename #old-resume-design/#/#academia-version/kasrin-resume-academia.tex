%% start of file `template.tex'.
%% Copyright 2006-2015 Xavier Danaux (xdanaux@gmail.com), 2020-2022 moderncv maintainers (github.com/moderncv).
%
% This work may be distributed and/or modified under the
% conditions of the LaTeX Project Public License version 1.3c,
% available at http://www.latex-project.org/lppl/.


\documentclass[12pt,a4paper,sans]{moderncv}        % possible options include font size ('10pt', '11pt' and '12pt'), paper size ('a4paper', 'letterpaper', 'a5paper', 'legalpaper', 'executivepaper' and 'landscape') and font family ('sans' and 'roman')

\usepackage{xcolor}


% moderncv themes
\moderncvstyle{classic}                             % style options are 'casual' (default), 'classic', 'banking', 'oldstyle' and 'fancy'
\moderncvcolor{blue}                               % color options 'black', 'blue' (default), 'burgundy', 'green', 'grey', 'orange', 'purple' and 'red'
%\renewcommand{\familydefault}{\sfdefault}         % to set the default font; use '\sfdefault' for the default sans serif font, '\rmdefault' for the default roman one, or any tex font name
%\nopagenumbers{}                                  % uncomment to suppress automatic page numbering for CVs longer than one page

% adjust the page margins
\usepackage[scale=0.8]{geometry}
\setlength{\footskip}{120pt}                 % depending on the amount of information in the footer, you need to change this value. comment this line out and sinet it to the size given  the warning
%\setlength{\hintscolumnwidth}{3cm}                % if you want to change the width of the column with the dates
%\setlength{\makecvheadnamewidth}{10cm}            % for the 'classic' style, if you want to force the width allocated to your name and avoid line breaks. be careful though, the length is normally calculated to avoid any overlap with your personal info; use this at your own typographical risks...

% font loading
% for luatex and xetex, do not use inputenc and fontenc
% see https://tex.stackexchange.com/a/496643
\ifxetexorluatex
  \usepackage{fontspec}
  \usepackage{unicode-math}
  \defaultfontfeatures{Ligatures=TeX}
  \setmainfont{Latin Modern Roman}
  \setsansfont{Latin Modern Sans}
  \setmonofont{Latin Modern Mono}
  \setmathfont{Latin Modern Math} 
\else
  \usepackage[utf8]{inputenc}
  \usepackage[T1]{fontenc}
  \usepackage{lmodern}
\fi

% document language
\usepackage[english]{babel} 

% personal data
\name{Nasr}{Kasrin}
\title{{\normalsize Data, Information, \& Knowledge Management}}
%\born{January 1986}                                 
\address{Hegelstr. 8, 96052}{Bamberg, Germany}{\textcolor{blue}{\textbf{Permanent Resident}}}
\phone[mobile]{+49~176~36642113}                 
%\phone[fixed]{+2~(345)~678~901}
%\phone[fax]{+3~(456)~789~012}
\email{nkasrin@gmail.com}                               
%\homepage{https://nskss.github.io/}                    

% Social icons
%\social[linkedin]{nkasrin}                         % optional, remove / comment the line if not wanted
%\social[xing]{john\_doe}                           % optional, remove / comment the line if not wantedted
\social[github]{nskss}                              % optional, remove / comment the line if not wanted
%\social[researchgate]{jdoe}                        % optional, remove / comment the line if not wanted
%\social[researcherid]{jdoe}                        % optional, remove / comment the line if not wanted
%%\social[matrix]{@johndoe:matrix.org}               % optional, remove / comment the line if not wanted
%\social[googlescholar]{JgS4-1cAAAAJ}            % optional, remove / comment the line if not wanted


%\extrainfo{additional information}                 % optional, remove / comment the line if not wanted
\photo[64pt][0.4pt]{picture}                       % optional, remove / comment the line if not wanted; '64pt' is the height the picture must be resized to, 0.4pt is the thickness of the frame around it (put it to 0pt for no frame) and 'picture' is the name of the picture file
%\quote{Some quote}                                 % optional, remove / comment the line if not wanted

% bibliography adjustments (only useful if you make citations in your resume, or print a list of publications using BibTeX)
%   to show numerical labels in the bibliography (default is to show no labels)
%\makeatletter\renewcommand*{\bibliographyitemlabel}{\@biblabel{\arabic{enumiv}}}\makeatother
\renewcommand*{\bibliographyitemlabel}{[\arabic{enumiv}]}
%   to redefine the bibliography heading string ("Publications")
%\renewcommand{\refname}{Articles}

\nopagenumbers{}

% bibliography with mutiple entries
%\usepackage{multibib}
%\newcites{book,misc}{{Books},{Others}}
%----------------------------------------------------------------------------------
%            content
%----------------------------------------------------------------------------------
\begin{document}
%\begin{CJK*}{UTF8}{gbsn}                          % to typeset your resume in Chinese using CJK
%-----       resume       ---------------------------------------------------------
\makecvtitle

\section{About}

Curious, self-motivated, and empathetic--a research scientist by training and a product developer by practice. From the technical side, I tend to gravitate to the architectural side of things (recently, cross-system interfaces in my Ph.D.). I am productive in positions that mix technical and non-technical work. I also have a track record in leadership positions.

%These experiences met in my doctoral work, a formal model for sharing and exchanging data across technology infrastructures. I am productive in positions that mix technical and non-technical work. I also have a track record in leadership and managerial positions.


%
%\section{\emph{At a Glance}}
%
%\cventry{2021--2022}{Doctoral Student}{University of Bamberg}{}{Germany}{}
%
%\cventry{2015--2020}{Researcher / Product Manager / Software Architect}{University of Bamberg}{}{}{}
%
%\cventry{2013--2015}{Acting Dean / Lecturer}{International University of Technology Twintech}{Yemen}{}{}
%
%\cventry{2011--2013}{Team Leader / Product Manager / Software Architect / R\&D}{TayaIT}{Egypt}{}{}
%
%\cventry{2008--2012}{Teaching Assistant}{German University in Cairo (GUC)}{Egypt}{}{}
%



%\section{Master thesis}
%\cvitem{title}{\emph{Title}}
%\cvitem{supervisors}{Supervisors}
%\cvitem{description}{Short thesis abstract}

%\section{Experience}

\section{Product Development / R\&D}

%\emph{Experience separated into categories. Overlaps denote multiple concurrent roles.}

%\subsection{R\&D / Product Development / Managerial}
\cventry{2015 -- 2021}{Researcher Associate}{University of Bamberg (Third-Party Projects)}{}{Germany}{Designed and delivered data management solutions for a computer-aided manufacturing project which included companies such as Airbus (explorative/R\&D project).}


\cventry{2011 -- 2013}{Team Leader / Software Architect / R\&D Engineer}{TayaIT}{Cairo}{Egypt}{Led the development of several iOS apps and games as product owner (reported to the CEO), built architectures, computer (AI) players; responsible for aligning the technical, marketing, and strategy teams.}



\section{Academic / Managerial}


\cventry{2013 -- 2015}{Lecturer / Managerial Roles}{University of Technology Twintech}{}{Yemen}{Carried out managerial duties (reporting to the university president) and taught several courses (ex., Object Oriented Programming, Art \& Design History, Human-Computer Interaction).}


\cventry{2008 -- 2012}{Teaching Assistant}{The German University in Cairo (GUC)}{Cairo}{Egypt}{Taught 10 hours per week of tutorials, in addition to grading and exam correction (courses: Introduction to Artificial Intelligence, Introduction to Computer Science, Introduction to Computer Programming).}

\section{Education}
\cventry{2019 -- 2023}{Ph.D.}{The University of Bamberg}{Germany (\emph{Prospective})}{}{}%{Thesis: The Basin Network: A Model for Sharing and Exchanging Data}  % arguments 3 to 6 can be left empty
\cventry{2009 -- 2010}{MSc.}{Faculty of Engineering, The German University in Cairo (GUC)}{Egypt}{}{}%{Thesis: Multi-modal Perception as Grounded Focussed Belief Revision}
\cventry{2003 -- 2008}{BSc.}{Faculty of Engineering, The German University in Cairo (GUC)}{Egypt}{}{}








\newpage








\section{Technical}
\begin{cvcolumns}
  \cvcolumn{Good Familiarity}{Python, Neo4j, Prolog, git, \LaTeX, Linux CLI, Lisp, Java (v5)}
  \cvcolumn{Past Experience}{SQL, MongoDB, UML, Redis, Pearl, C++, Lucene, Micro-C programming}
\end{cvcolumns}

\vspace{-3.5ex}

%\cvdoubleitem{category 2}{XXX, YYY, ZZZ}{category 5}{XXX, YYY, ZZZ}
%\cvdoubleitem{category 3}{XXX, YYY, ZZZ}{category 6}{XXX, YYY, ZZZ}

\section{Projects}
%Projects I was involved in one or more capacities: product owner, architect, organizer, and developer.  
%\cvitemwithcomment{2016 -- 2022}{\textbf{Basin Net}: A model for data exchange across organizations.}{PhD project}
\cvitemwithcomment{2017 -- 2020}{\textbf{kgservice}: A semantic graph database interface.}{\textcolor{blue}{\href{https://github.com/simutool/kgservice}{\faGithub/simutool/kgservice}}}{}
\cvitemwithcomment{2019}{\textbf{aku-client}: Meta-data client of the kgservice.}{\textcolor{blue}{\href{https://github.com/simutool/aku-client}{\faGithub/simutool/aku-client}}}{}
\cvitemwithcomment{2018}{\textbf{om-tool}: A tool to visualize and enrich sensor data.}{\textcolor{blue}{\href{https://github.com/simutool/om-tool}{\faGithub/simutool/om-tool}}}{}
%\cvitemwithcomment{2017}{\textbf{dm-reader}: Maps an ontology from excel to python dicts.}{\textcolor{blue}{\href{https://github.com/simutool/dm-reader}{\faGithub/simutool/dm-reader}}}{}
\cvitemwithcomment{2017}{\textbf{model-builder}: Loads an ontology to a Neo4j graph.}{\textcolor{blue}{\href{https://github.com/simutool/model-builder}{\faGithub/simutool/model-builder}}}{}
\cvitemwithcomment{2005 -- 2015}{A detailed portfolio for this period can be retrieved from:}{\textcolor{blue}{\href{https://bit.ly/3HTuluG}{bit.ly/3HTuluG}}}

\section{Languages}
\begin{cvcolumns}
  \cvcolumn{English}{Fluent}
  \cvcolumn{Arabic}{Fluent}
  \cvcolumn{German}{B1 (Telc Certificate)}  
\end{cvcolumns}

\vspace{-3.5ex}


% Publications from a BibTeX file without multibib
%  for numerical labels: \renewcommand{\bibliographyitemlabel}{\@biblabel{\arabic{enumiv}}}% CONSIDER MERGING WITH PREAMBLE PART
%  to redefine the heading string ("Publications"): 
\renewcommand{\refname}{Publications}
\nocite{*}
\bibliographystyle{unsrt}
\bibliography{publications}                        % 'publications' is the name of a BibTeX file


\section{References}
\cvitemwithcomment{}{\textbf{Daniela Nicklas}, \textit{Professor}, University of Bamberg}{}
\cvitemwithcomment{}{\textbf{Haythem Ismail}, \textit{Dean}, MET Faculty, The German University in Cairo}{}
\cvitemwithcomment{}{\textbf{Ashraf Tawwakol}, \textit{Co-founder}, Intersection; formerly, \emph{CEO}, TayaIT}{}
\cvitemwithcomment{}{\textbf{Wael Alaghbari}, \textit{President}, International University of Technology Twintech}{}









% Publications from a BibTeX file using the multibib package
%\section{Publications}
%\nocitebook{book1,book2}
%\bibliographystylebook{plain}
%\bibliographybook{publications}                   % 'publications' is the name of a BibTeX file
%\nocitemisc{misc1,misc2,misc3}
%\bibliographystylemisc{plain}
%\bibliographymisc{publications}                   % 'publications' is the name of a BibTeX file

%\clearpage
%%-----       letter       ---------------------------------------------------------
%% recipient data
%\recipient{Company Recruitment team}{Company, Inc.\\123 somestreet\\some city}
%\date{January 01, 1984}
%\opening{Dear Sir or Madam,}
%\closing{Yours faithfully,}
%\enclosure[Attached]{curriculum vit\ae{}}          % use an optional argument to use a string other than "Enclosure", or redefine \enclname
%\makelettertitle
%
%Lorem ipsum dolor sit amet, consectetur adipiscing elit. Duis ullamcorper neque sit amet lectus facilisis sed luctus nisl iaculis. Vivamus at neque arcu, sed tempor quam. Curabitur pharetra tincidunt tincidunt. Morbi volutpat feugiat mauris, quis tempor neque vehicula volutpat. Duis tristique justo vel massa fermentum accumsan. Mauris ante elit, feugiat vestibulum tempor eget, eleifend ac ipsum. Donec scelerisque lobortis ipsum eu vestibulum. Pellentesque vel massa at felis accumsan rhoncus.
%
%Suspendisse commodo, massa eu congue tincidunt, elit mauris pellentesque orci, cursus tempor odio nisl euismod augue. Aliquam adipiscing nibh ut odio sodales et pulvinar tortor laoreet. Mauris a accumsan ligula. Class aptent taciti sociosqu ad litora torquent per conubia nostra, per inceptos himenaeos. Suspendisse vulputate sem vehicula ipsum varius nec tempus dui dapibus. Phasellus et est urna, ut auctor erat. Sed tincidunt odio id odio aliquam mattis. Donec sapien nulla, feugiat eget adipiscing sit amet, lacinia ut dolor. Phasellus tincidunt, leo a fringilla consectetur, felis diam aliquam urna, vitae aliquet lectus orci nec velit. Vivamus dapibus varius blandit.
%
%Duis sit amet magna ante, at sodales diam. Aenean consectetur porta risus et sagittis. Ut interdum, enim varius pellentesque tincidunt, magna libero sodales tortor, ut fermentum nunc metus a ante. Vivamus odio leo, tincidunt eu luctus ut, sollicitudin sit amet metus. Nunc sed orci lectus. Ut sodales magna sed velit volutpat sit amet pulvinar diam venenatis.
%
%Albert Einstein discovered that $e=mc^2$ in 1905.
%
%\[ e=\lim_{n \to \infty} \left(1+\frac{1}{n}\right)^n \]
%
%\makeletterclosing



\end{document}


%% end of file `template.tex'.

