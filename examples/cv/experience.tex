%-------------------------------------------------------------------------------
%	SECTION TITLE
%-------------------------------------------------------------------------------
\cvsection{Expérience}


%-------------------------------------------------------------------------------
%	CONTENT
%-------------------------------------------------------------------------------
\begin{cventries}

	\cventry
	{Ingénieur logiciel embarqué} % Intitulé du poste
	{SII, en mission chez MBDA} % Organisation
	{Le Plessis-Robinson, France} % Lieu
	{Février 2024 - Présent} % Dates
	{
		\begin{cvitems} % Description(s) des tâches/responsabilités
		\item{
			Projet : Développement de plusieurs logiciels embarqués en \textbf{C++} et \textbf{Java}, utilisant \textbf{IBM Rational Rhapsody} pour la génération de code
			\begin{itemize}
					\item{Ces logiciels communiquent via un réseau \textbf{Ethernet} et un bus \textbf{1553}}
			\end{itemize}
		}
		\item{
			\textbf{Département Ingénierie Logicielle (\emph{SE}} : Membre d'une équipe de 4 développeurs, avec pour principales responsabilités :
			\begin{itemize}
				\item{\textbf{C++}, \textbf{Java} : Conception et développement de nouvelles fonctionnalités, refonte de fonctionnalités existantes issues de code \emph{legacy}, correction de bugs existants}
				\item{Tests d'intégration des logiciels sur \textbf{bancs de test} : utilisation de moyens de test (simulateurs propriétaires) pour tester le système}
				\item{Génération personnalisée de \textbf{Linux Embarqué}}
				\item{Recherche et compilation de \textbf{drivers} mis à jour (modules noyau) pour de nouvelles versions de noyaux Linux}
				\item{Portage de \textbf{drivers} non mis à jour vers de nouveaux noyaux Linux (drivers écrits en \textbf{C})}
				\item{Portage du code source vers une nouvelle version du compilateur (\textbf{gcc}) : résolution des problèmes de compilation dus aux évolutions des standards \textbf{C++}}
			\end{itemize}
		}
		\item{
			Travail réalisé de manière autonome et à mon initiative, pour \textbf{améliorer les processus de travail et la qualité des livrables du projet} :
			\begin{itemize}
				\item{Amélioration de la configuration réseau (\textbf{TCP/IP}) du système}
				\item{Utilisation de langages de script : \textbf{Python}, \textbf{PowerShell}, \textbf{bash} et \textbf{Makefile} pour l'automatisation des tests, de la compilation et de la documentation}
				\item{Automatisation de \textbf{IBM Rational Rhapsody} via l'API \textbf{Java}}
				\item{Automatisation de \textbf{Wireshark} via l'interface en ligne de commande \textbf{tshark}}
				\item{Écriture d'un générateur de code \textbf{C} en Python pour décoder des payloads réseau en structures \textbf{C}. \textbf{C++} : Génération de tests unitaires aléatoires pour le code généré}	
				\item{Développement de scripts \textbf{bash} très robustes pour la synchronisation temporelle \textbf{NTP}, afin de résoudre des problèmes de synchronisation récurrents}
			\end{itemize}
		}
		\end{cvitems}
	}

	\cventry
	{Ingénieur logiciel embarqué et simulation (contrat de thèse)} % Intitulé du poste
	{Laboratoire TIMA} % Organisation
	{Grenoble, France} % Lieu
	{Octobre 2022 - Janvier 2024} % Dates
	{
		\begin{cvitems} % Description(s) des tâches/responsabilités
		\item{
			Modélisation et simulation d'environnements de systèmes de contrôle :
			\begin{itemize}
				\item{Évaluation comparative de la simulation de systèmes cyber-physiques dans \textbf{Simulink} et \textbf{SystemC}}
			\end{itemize}
		}
		\item{
			Ingénierie et développement logiciel :
			\begin{itemize}
				\item{Modèles \textbf{Simulink/MATLAB} (blocs personnalisés utilisant \textbf{S-Function} en \textbf{C})}
				\item{Modèles \textbf{SystemC (C++)} (AMS/TLM) de SoC (\textbf{System-On-Chip})}
				\item{Restauration et refonte de code \emph{legacy} en \textbf{C++}, refonte de \textbf{Makefile}s}
			\end{itemize}
		}
		\end{cvitems}
	}

	\cventry
	{Enseignant} % Intitulé du poste
	{Université Grenoble Alpes, DLST} % Organisation
	{Grenoble, France} % Lieu
	{Janvier 2023 - Mai 2023} % Dates
	{
		\begin{cvitems} % Description(s) des tâches/responsabilités
		\item{Cours « Systèmes et environnement de programmation » pour étudiants de première année : travaux pratiques et exercices en classe}
		\item{\textbf{Scripts Bash}, bases de la programmation en \textbf{C} et modélisation simple par machines à états finis}
		\end{cvitems}
	}

	%---------------------------------------------------------
	\cventry
	{Stagiaire Master 2} % Intitulé du poste
	{Laboratoire TIMA} % Organisation
	{Grenoble, France} % Lieu
	{Février 2022 - Juin 2022} % Dates
	{
		\begin{cvitems} % Description(s) des tâches/responsabilités
		\item{Simulation parallèle de modèles de systèmes cyber-physiques/embarqué en \textbf{SystemC (C++)}}
		\item{Restauration et refonte de code \emph{legacy} \textbf{SystemC (C++)} et de \textbf{Makefile}s	}		
		\end{cvitems}
	}

	%---------------------------------------------------------
	\cventry
	{Stagiaire Assistant Ingénieur} % Intitulé du poste
	{Kayentis} % Organisation
	{Meylan, France} % Lieu
	{Mai 2021 - Septembre 2021} % Dates
	{
		\begin{cvitems} % Description(s) des tâches/responsabilités
		\item{Développement d'un script \textbf{PowerShell} pour un système de réponse automatique aux fautes}
		\item{Automatisation de navigateur web avec \textbf{Python Selenium} pour vérifier le fonctionnement d'une application Web}
		\item{Utilisation de \textbf{Jenkins} pour la surveillance automatique de l'application Web}
		\end{cvitems}
	}

	\cventry
	{Stagiaire} % Intitulé du poste
	{Givaudan} % Organisation
	{Vernier, Suisse} % Lieu
	{Avril 2019 - Juillet 2019} % Dates
	{
		\begin{cvitems} % Description(s) des tâches/responsabilités
		\item{Recherche et déploiement d'une solution de surveillance de réseau \textbf{Ethernet} (Advanced Host Monitor)}
		\end{cvitems}
	}

	\cventry
	{Stagiaire} % Intitulé du poste
	{Abissa Informatique Genève} % Organisation
	{Genève, Suisse} % Lieu
	{Juillet 2018 - Août 2018} % Dates
	{
		\begin{cvitems} % Description(s) des tâches/responsabilités
		\item{Recherche et déploiement d'une solution de gestion d'adresses IP (IPAM) (PhpIpam)}
		\end{cvitems}
	}
	%---------------------------------------------------------
\end{cventries}
