%-------------------------------------------------------------------------------
%	SECTION TITLE
%-------------------------------------------------------------------------------
\cvsection{Professional Experience}



%-------------------------------------------------------------------------------
%	CONTENT
%-------------------------------------------------------------------------------
\begin{cventries}


%---------------------------------------------------------


\cventry
    {Senior Data Scientist II} % Job title
    {Credit Karma} % Organization
    {San Diego, CA} % Location
    {Jun. 2022 - Present} % Date(s)
    {\begin{cvitems}
        \setlength\itemsep{.15em}
        \item { \textbf{Mint Migration: } As the sole data scientist to move to Credit Karma from the Mint acquistion, my responsibilities included sunsetting the Mint AI assets while maintaining revenue and MAU and developing and executing the technical roadmap  to launch Net Worth features  shifting Credit Karma from a purely free credit score app to a digital financial assistant.  }
        \item { \textbf{CK Transaction Categorization: } Created the new version of the transaction categorization for the app that simplied categorization for personal finance compared to the Mint app. Categorized transactions are used to surface personal finance insights for customers through the CK Accounts Tab, Intuit Assist LLM, email and push notifications leading to increased engagment and customer LTV.}
         \item { \textbf{Recurring Transactions Identification: } Created a new feature at CK to identify recurring transactions such as subscriptions and bills to help members better understand how they are spending.}
         \item { \textbf{Credit Karma 2 TurboTax:} Responsible for creating ML-models for finding customers that have propensity to use the CK2TT product as well as creating models to predict and pre-fill tax forms for making the tax process easier for our members. }
     \end{cvitems}}


\cventry
    {Staff Data Scientist} % Job title
    {Intuit} % Organization
    {San Diego, CA} % Location
    {Jun. 2019 - Jun. 2022} % Date(s)
    {\begin{cvitems}
        \setlength\itemsep{.15em}
        \item { \textbf{Overdraft Early Warning System (Mint): }Built and deployed a machine learning system to predict when customers are likely to overdraft and prevent the overdraft through an email notification. This feature saved Mint customers 12 million dollars in a single year and was highlighted in both CNET and The Wall Street Journal as well as accepted to the \href{https://dl.acm.org/doi/10.1145/3637528.3671628}{\underline{KDD 2024 ADS Track}}. }
        \item { \textbf{Transaction Categorization (Mint): }  Revamped the current Mint Transaction Categorization by developing a new BERT-based model. This model enabled the first Venmo and Zelle support as well as increased the overall accuracy of the model. The overall accuracy led to the overall increase of usability of features that rely on categorized transactions. }
         \item { \textbf{Forecasting for Personal Finance (Mint): } Led a team of three data scientists to develop an RNN-based forecasting model to forecast future spending of Mint customers. This feature increased sign-ups for (new sunsetted) the Mint Premium (paid) tier.}
         \item { \textbf{Dynamic Upsell (TurboTax):} Built a model to select if a customer should be shown an upsell or not in TurboTax. The model decreased churn due to upsell fatigue by suppressing upsells for customers that are the least likely to be upselled. }
     \end{cvitems}}


\cventry
    {Data Scientist/Research Scientist} % Job title
    {Center for Data Science \& Public Policy, The University of Chicago} % Organization
    {Chicago,IL} % Location
    {May. 2016 - Jun. 2019} % Date(s)
    {\begin{cvitems}
        \setlength\itemsep{.15em}
        \item { \textbf{Retention of HIV Patients in Medical Care (Healthcare): }Created a risk assessment tool to predict which HIV+ patients are likely to drop out of care for use by the University of Chicago HIV clinic and Chicago Department of Public Health. This work was published in \href{https://www.nature.com/articles/s41598-020-62729-x}{\underline{Nature: Scientific Reports}}.}
        \item { \textbf{Prevention of Childhood Lead Poisoning (Healthcare): }Deployed a machine learning model for predicting which homes in the city of Chicago are likely to have lead hazards that lead to early childhood lead poisoning. This work was in  partnership with the Chicago Department of Public Health, the Chicago Department of Innovation and Technology and Alliance HealthCare, awarded the Academy Health Local/State Innovation Prize.}
         \item { \textbf{Preventing Water Main Breaks (City Planning): }Built and deployed a machine learning system to identify which city blocks are most at risk of having water main breaks for the city of Syracuse, NY. This work has been featured in State Scoop, Water Online, and Politico and replicated in several cities as well as published in \href{https://dl.acm.org/doi/abs/10.1145/3219819.3219835}{\underline{KDD ADS 2018}}.}
         \item { \textbf{The Cost of Technical Recidivism (Criminal Justice): }Developed a point-of-service model for predicting general recidivism for the Illinois Department of Corrections. Conducted a retrospective study to understand the effect of technical recidivism violations on an offender's ability to obtain future employment.}
         \item{ \textbf{Executive Training: }Wrote and taught the curriculum for the Coleridge Initiative, a 3-month long course to train the heads of city and state government agencies on the use of data science methods to solve public policy problems.}
     \end{cvitems}}



%---------------------------------------------------------
  \cventry
    {Postdoctoral Research Associate/Systems Administrator} % Organization
    {Arizona State University, P.I. Associate Professor S. Banu Ozkan} % Job title
    {Tempe, AZ} % Location
    {August 2014 - September 2016} % Date(s)
    {
      \begin{cvitems} % Description(s) of tasks/responsibilities
        \item {Worked on problems related to protein dynamics, protein structure refinement, genetic disease prediction, and antibiotic resistance using molecular dynamics and machine learning methods.}
        \item {Wrote several software packages for studying protein dynamics and analyzing genetic disease in Python.}
        \item {Built and maintained a 1408 node supercomputer as systems administrator.}
        \item {Authored six publications in peer reviewed journals and mentored two doctoral students.}
      \end{cvitems}
    }

%---------------------------------------------------------
  \cventry
  {Graduate Research Assistant} % Job title
  {Arizona State University, P.I. Professor Michael F. Thorpe} % Organization
  {Tempe, AZ} % Location
  {September 2009 - August 2014} % Date(s)
    {
      \begin{cvitems} % Description(s) of tasks/responsibilities
        \item {Developed software packages in C++, Python, and Fortran to study amorphous materials.}
        \item {Authored five publications in peer reviewed journals. Awarded over \$60,000 through multiple fellowships to fund research.}
      \end{cvitems}
    }


\end{cventries}
