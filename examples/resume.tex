%!TEX TS-program = xelatex
%!TEX encoding = UTF-8 Unicode
% Awesome CV LaTeX Template for CV/Resume
%
% This template has been downloaded from:
% https://github.com/posquit0/Awesome-CV
%
% Author:
% Claud D. Park <posquit0.bj@gmail.com>
% http://www.posquit0.com
%
% Template license:
% CC BY-SA 4.0 (https://creativecommons.org/licenses/by-sa/4.0/)
%

%-------------------------------------------------------------------------------
% CONFIGURATIONS
%-------------------------------------------------------------------------------
% A4 paper size by default, use 'letterpaper' for US letter
\documentclass[11pt, a4paper]{awesome-cv}
\usepackage{fontspec}
\usepackage{fontawesome}
\usepackage{xltxtra}

% Configure page margins with geometry
\geometry{left=1.4cm, top=.8cm, right=1.4cm, bottom=1.8cm, footskip=.5cm}

% Specify the location of the included fonts
\fontdir[fonts/]

% Color for highlights
% Awesome Colors: awesome-emerald, awesome-skyblue, awesome-red, awesome-pink, awesome-orange
%                 awesome-nephritis, awesome-concrete, awesome-darknight
\colorlet{awesome}{awesome-red}
% Uncomment if you would like to specify your own color
% \definecolor{awesome}{HTML}{3E6D9C}

% Colors for text
% Uncomment if you would like to specify your own color
\definecolor{darktext}{HTML}{414141}
% \definecolor{text}{HTML}{333333}
% \definecolor{graytext}{HTML}{5D5D5D}
% \definecolor{lighttext}{HTML}{999999}
% \definecolor{sectiondivider}{HTML}{5D5D5D}

% Set false if you don't want to highlight section with awesome color
\setbool{acvSectionColorHighlight}{true}

% If you would like to change the social information separator from a pipe (|) to something else
\renewcommand{\acvHeaderSocialSep}{\quad\textbar\quad}


%-------------------------------------------------------------------------------
%	PERSONAL INFORMATION
%	Comment any of the lines below if they are not required
%-------------------------------------------------------------------------------
% Available options: circle|rectangle,edge/noedge,left/right
\photo[left]{../../../images/grum.png}
\name{Romain}{Gallet}
\position{Software Architecture{\enskip\cdotp\enskip}Data engineering}
%\address{42-8, Bangbae-ro 15-gil, Seocho-gu, Seoul, 00681, Rep. of KOREA}

\cpyname{Grum}{Ltd}

\mobile{(+44) 787 243 6423 }
\email{info@grumlimited.co.uk}
\homepage{grumlimited.co.uk}
\github{grumlimited}
\linkedin{romaingallet}
% \gitlab{gitlab-id}
% \stackoverflow{SO-id}{SO-name}
% \twitter{@twit}
% \skype{skype-id}
% \reddit{reddit-id}
% \medium{madium-id}
% \kaggle{kaggle-id}
% \hackerrank{hackerrank-id}
% \googlescholar{googlescholar-id}{name-to-display}
%% \firstname and \lastname will be used
% \googlescholar{googlescholar-id}{}
% \extrainfo{extra information}

\quote{ %\textcolor{awesome}{\textbf{CONTRACTS ONLY - LONDON ONLY}}\\
\vspace{1em}
Engineering and consultancy services within the media, financial and publishing industries.\\
\textbf{Grum Ltd} provides expertise in high availability architectures around streaming data pipelines.}

\hypersetup{%
  pdftitle={Grum Limited~~~·~~~Consultancy},
  pdfauthor={Romain Gallet},
  pdfsubject={Grum Limited~~~·~~~Consultancy},
  pdfkeywords={CV, Grum, Consultancy}
}

%-------------------------------------------------------------------------------
\begin{document}

% Print the header with above personal information
% Give optional argument to change alignment(C: center, L: left, R: right)
\makecvheader[C]

% Print the footer with 3 arguments(<left>, <center>, <right>)
% Leave any of these blank if they are not needed
\makecvfooter
  %{\today}
  {}
  {Grum Limited~~~·~~~Consultancy}
  {\thepage}


%-------------------------------------------------------------------------------
%	CV/RESUME CONTENT
%	Each section is imported separately, open each file in turn to modify content
%-------------------------------------------------------------------------------
%%-------------------------------------------------------------------------------
%	SECTION TITLE
%-------------------------------------------------------------------------------
\cvsection{Summary}


%-------------------------------------------------------------------------------
%	CONTENT
%-------------------------------------------------------------------------------
\begin{cvparagraph}

%---------------------------------------------------------
Current Co-founder \& Software Engineer in start-up company PLAT Corp. 5+ years experience specializing in backend/infrastructure, web development and computer security. Super nerd who loves Vim, Linux and OS X and enjoys to customize all of the development environment. Interested in devising a better problem-solving method for challenging tasks, and learning new technologies and tools if the need arises.
\end{cvparagraph}

%-------------------------------------------------------------------------------
%	SECTION TITLE
%-------------------------------------------------------------------------------
\cvsection{Experience}


%-------------------------------------------------------------------------------
%	CONTENT
%-------------------------------------------------------------------------------
\begin{cventries}

%---------------------------------------------------------
  \cventry
    {Senior Technical Lead} % Job title
    {Bellroy (TrikeApps was merged into Bellroy in Apr. 2018)} % Organization
    {Melbourne, Australia} % Location
    {Nov. 2017 - Present} % Date(s)
    {
      \begin{cvitems} % Description(s) of tasks/responsibilities
        \item {Actively participating in architectural discussion with Senior Devs to clarify and document architectural goals.}
        \item {Guide juniors on individual stories to provide big picture and ensure code quality.}
        \item {Regularly perform code reviews to encourage best prctices.}
        \item {Lead EDI (Electronic Data Interchange) integration efforts for multiple high-profile clients.}
        \item {Built ETL pipelines for syncing orders from digital marketplaces. Wrote a parser to generate rails migrations from API spec for core business entities of interest to optimise the time spent to build future ETL pipelines.}
        \item {Supported the business with technical requiremnts for a business entity restructure.}
        \item {Wrote a caching layer for a throttled external API which has saved > 5M api calls (171 days worth of api calls). Which has saved systems from being blocked countless times.}
        \item {Wrote a Background job throttler which can automaically filter low value background jobs and defer them to next day (while it was running low on external API calls for the day) to be retried automatically so ciritical business processes could continue running. This saved a lot of tech support time.}
        \item {Worked with a colleague to plan and move multiple self-hosted postgres databases into AWS aurora without much downtime. Performed mental simulations and upfront analysis to cut down disruptions of business processes.}
        \item {Implemented a generalised shipping alert rules which allowed Logistcs to flag shipments that matched certain criteries to enforce better business processes.}
        \item {Participated in and heavily contributed to a massive refactoring of a order fulfilment pipeline, following clean architecture practies, which led to the system processing 200K+ background jobs without a single failure on a recent sale spike in 24 hours.}
        \item {In an effort to build a CI/CD pipeline on AWS, migrated CI from Travis to AWS CodeBuild. Used AWS CloudFormation to build the stack (including Docker images). It also cut down cost by ~70\% while also cutting down spec run time for our largest project by several minutes. Also Built an Elm component for internal dashboard to show CodeBuild build statues.}
        \item {Spent peronsal time to build a demo CD pipeline from scratch on EKS using first Terraform , then cloudformation with Blue/Green deploys which contributed to the approval of CI/CD project.}
        \item {Paired with another dev to build a staging Kuberntes Cluster on EKS using cloudformation, following good networking and security practices to progress on the CI/CD project.}
        \item {Spent personal time to build Grafana dashboard and add hubot (with slack integration) to encourage DevOps/chatops practices.}
        \item {Leading data migration effort of an undergoing ERP integration project.}
      \end{cvitems}
    }
%---------------------------------------------------------
  \cventry
    {Team Lead} % Job title
    {TrikeApps} % Organization
    {Melbourne, Australia} % Location
    {Apr. 2017 - Oct. 2017} % Date(s)
    {
      \begin{cvitems} % Description(s) of tasks/responsibilities
        \item {Contributed to creating an internal DSL for specifying shipping rules in primary client's e-commerce store which allowed specifying complex shipping rules.}
        \item {Prepared internal systems to be able to work with (with the accounting system integration) a business entity change for primary client.}
        \item {Managed 2 other Junior devs, participated in sprint planning and estimation and organized retrospecives.}
        \item {Took the responsibility of meeting with Transactions team weekly, and worked closely with them to maintain and support the rapid growth of the business.}
        \item {Integrated multiple new digital marketplaces, including automated order fulfilment, settlement/payment report processing etc.}
      \end{cvitems}
    }
%---------------------------------------------------------
  \cventry
    {Ruby Developer} % Job title
    {TrikeApps} % Organization
    {Melbourne, Australia} % Location
    {Apr. 2016 - Apr. 2017} % Date(s)
    {
      \begin{cvitems} % Description(s) of tasks/responsibilities
        \item {Upgraded multiple internal apps to Rails 5.}
        \item {Added Apple Pay Payment method on primary client's e-commerce store which significantly increased conversion from qualifying Safari users.}
        \item {Built a S3 backed drag-n-drop product image management system for primary client's e-commerce store.}
        \item {Converted a legacy ruby app into a Rails one for consistency.}
        \item {Wrote a data/business validity checker which let others write SQL queries to run periodically and report them on business ciritical processes without help from tech support.}
        \item {Built and maintained a self-hosted analytics platform (Snowplow) for the data analytics team, involved learning about, setting up and maintaining the full stack including AWS CloudFront, Hadoop/Spark jobs, AWS Athena, AWS Redshift. This supported high performance, low overhead batch event processing for all systems.}
      \end{cvitems}
    }
%---------------------------------------------------------
  \cventry
    {PHP Software Developer} % Job title
    {Astute Payroll} % Organization
    {Melbourne, Australia} % Location
    {Feb. 2016 - Mar. 2017} % Date(s)
    {
      \begin{cvitems} % Description(s) of tasks/responsibilities
        \item {Refactored the document management system from legacy code to S3 backed clean component using Twig templates.}
        \item {Refactored payslip rendering service to a domain service which was then used in cli, API and user facing features.}
        \item {Wrote a robust regression tester script in PHP which could generate thousands of payslips using old and new renderer and report on differences to build confidence before cutover.}
        \item {Worked on third party integrations.}
      \end{cvitems}
    }

%---------------------------------------------------------
  \cventry
    {Web Developer} % Job title
    {Adgate Media} % Organization
    {Remote} % Location
    {Feb. 2015 - Jan. 2016} % Date(s)
    {
      \begin{cvitems} % Description(s) of tasks/responsibilities
        \item {Implemented complex ideas into working solutions using Laravel.}
        \item {Used OOD best practices to produce testable functional code.}
        \item {Learnt about Unit and functional testing and started following TDD.}
        \item {Built a multi-room chat functionality with ReactPHP on the backend and ReactJS on the frontend with Websocket and ZeroMQ (with support for custom commands).}
        \item {Added real-time frontend update support via websocket, to deliver critical information without page refresh.}
        \item {Built cache decorators with repository patterns to make caching layer transparent.}
        \item {Used AWS SQS for background job processing.}
        \item {Built a permission system which allowed feature restriction on user level.}
      \end{cvitems}
    }

%---------------------------------------------------------
  \cventry
    {Software Engineer} % Job title
    {Noobis Inc} % Organization
    {Dhaka, Bangladesh} % Location
    {Jul. 2011 - Sep. 2015} % Date(s)
    {
      \begin{cvitems} % Description(s) of tasks/responsibilities
        \item {Analysed client specifications and project requrements, and turned them into functional solutions.}
        \item {Prepared regular progress reports.}
        \item {Updating and Maintaining a large Magento installation including operational maintenace (learnt a lot about linux server administration).}
        \item {Developed in-house apps with Laravel and AngularJS.}
        \item {Identified and communicated changing priorities and business objectives to other team members and distributed workload.}
        \item {Built MVPs for investor presentations.}
      \end{cvitems}
    }
%---------------------------------------------------------
  \cventry
    {Web Developer} % Job title
    {Allmoxy Inc} % Organization
    {Remote Part Time} % Location
    {July. 2012 - May. 2013} % Date(s)
    {
      \begin{cvitems} % Description(s) of tasks/responsibilities
        \item {Added various reports into the existing app which gave more insignt into the business.}
        \item {Extended the custom MVC the client was using with more features.}
        \item {Extented the frontend built with PrototypeJS with more features.}
        \item {Created a documentation framework which let the customer support add video walkthrough and in turn saved time on support ticket resolution.}
      \end{cvitems}
    }

%---------------------------------------------------------
  \cventry
    {Web Developer Intern} % Job title
    {Leevio Inc.} % Organization
    {Remote Part Time} % Location
    {Dec. 2009 - Jun. 2011 } % Date(s)
    {
      \begin{cvitems} % Description(s) of tasks/responsibilities
      \item {Learnt about VCS (subversion).}
      \item {Contributed to the development of inhouse and client apps.}
      \item {Built few mobile apps with Titanium.}
      \item {Learnt and used Zend and CodeIgniter PHP frameworks.}
      \end{cvitems}
    }
%---------------------------------------------------------
\end{cventries}

%-------------------------------------------------------------------------------
%	SECTION TITLE
%-------------------------------------------------------------------------------
\cvsection{Education}


%-------------------------------------------------------------------------------
%	CONTENT
%-------------------------------------------------------------------------------
\begin{cventries}

%---------------------------------------------------------
  \cventry
    {B.S. in Computer Science and Engineering} % Degree
    {Rajshahi University Of Engineering And Technology} % Institution
    {Rajshahi, Bangladesh} % Location
    {Feb 2007 - Aug 2011} % Date(s)
    {
      % \begin{cvitems} % Description(s) bullet points
      %   \item {Got a Chun Shin-Il Scholarship which is given to promising students in CSE Dept.}
      % \end{cvitems}
    }

%---------------------------------------------------------
\end{cventries}

\cvsection{Honors \& Awards}
\cvsubsection{International}
\begin{cvhonors}
  \cvhonor
    {Finalist}
    {DEFCON 22nd CTF Hacking Competition World Final}
    {Las Vegas, U.S.A}
    {2014}
  \cvhonor
    {Finalist}
    {DEFCON 21st CTF Hacking Competition World Final}
    {Las Vegas, U.S.A}
    {2013}
  \cvhonor
    {Finalist}
    {DEFCON 19th CTF Hacking Competition World Final}
    {Las Vegas, U.S.A}
    {2011}
  \cvhonor
    {6th Place}
    {SECUINSIDE Hacking Competition World Final}
    {Seoul, S.Korea}
    {2012}
\end{cvhonors}

\cvsubsection{Domestic}
\begin{cvhonors}
  \cvhonor
    {3rd Place}
    {WITHCON Hacking Competition Final}
    {Seoul, S.Korea}
    {2015}
  \cvhonor
    {Silver Prize}
    {KISA HDCON Hacking Competition Final}
    {Seoul, S.Korea}
    {2013}
\end{cvhonors}

%\cvsection{Presentation}
\begin{cventries}
  \cventry
    {Presenter for <DEFCON 20th : The way to go to Las Vegas>}
    {6th CodeEngn (Reverse Engineering Conference)}
    {Seoul, S.Korea}
    {Jul. 2012}
    {
      \begin{cvitems}
        \item {Introduced CTF(Capture the Flag) hacking competition and advanced techniques and strategy for CTF}
      \end{cvitems}
    }
  \cventry
    {Presenter for <Metasploit 101>}
    {6th Hacking Camp - S.Korea}
    {S.Korea}
    {Sep. 2012}
    {
      \begin{cvitems}
        \item {Introduced basic procedure for penetration testing and how to use Metasploit}
      \end{cvitems}
    }
\end{cventries}

%-------------------------------------------------------------------------------
%	SECTION TITLE
%-------------------------------------------------------------------------------
\cvsection{Writing}


%-------------------------------------------------------------------------------
%	CONTENT
%-------------------------------------------------------------------------------
\begin{cventries}

%---------------------------------------------------------
  \cventry
    {Founder \& Writer} % Role
    {A Guide for Developers in Start-up} % Title
    {Facebook Page} % Location
    {Jan. 2015 - PRESENT} % Date(s)
    {
      \begin{cvitems} % Description(s)
        \item {Drafted daily news for developers in Korea about IT technologies, issues about start-up.}
      \end{cvitems}
    }

%---------------------------------------------------------
  \cventry
    {Undergraduate Student Reporter} % Role
    {AhnLab} % Title
    {S.Korea} % Location
    {Oct. 2012 - Jul. 2013} % Date(s)
    {
      \begin{cvitems} % Description(s)
        \item {Drafted reports about IT trends and Security issues on AhnLab Company magazine.}
      \end{cvitems}
    }

%---------------------------------------------------------
\end{cventries}

%%-------------------------------------------------------------------------------
%	SECTION TITLE
%-------------------------------------------------------------------------------
\cvsection{Program Committees}


%-------------------------------------------------------------------------------
%	CONTENT
%-------------------------------------------------------------------------------
\begin{cvhonors}

%---------------------------------------------------------
  \cvhonor
    {Problem Writer} % Position
    {2016 CODEGATE Hacking Competition World Final} % Committee
    {S.Korea} % Location
    {2016} % Date(s)

%---------------------------------------------------------
  \cvhonor
    {Organizer \& Co-director} % Position
    {1st POSTECH Hackathon} % Committee
    {S.Korea} % Location
    {2013} % Date(s)

%---------------------------------------------------------
\end{cvhonors}

%%-------------------------------------------------------------------------------
%	SECTION TITLE
%-------------------------------------------------------------------------------
\cvsection{Extracurricular Activity}


%-------------------------------------------------------------------------------
%	CONTENT
%-------------------------------------------------------------------------------
\begin{cventries}

%---------------------------------------------------------
\cventry
  {Stanford University, Online} % Affiliation/role
  {\textit{Machine Learning}}
  {Grade: 87\%} % Location
  {} % Date(s)
  {
    \begin{cvitems} % Description(s) of experience/contributions/knowledge
      \item {11 week course covering go to methods applied in supervised and unsupervised learning, the implementation in octave as well as topics like regularization and how to tackle problems like bias and overfitting.}
      \item {Certificate of accomplishment: http://tinyurl.com/v5ytdzhs}
    \end{cvitems}
  }
  \newline
%---------------------------------------------------------
\cventry
  {Kaggle.com, Online Machine-Learning Community} % Affiliation/role
  {\textit{Intro to Machine Learning}}
  {} % Location
  {2022} % Date(s)
  {
    \begin{cvitems} % Description(s) of experience/contributions/knowledge
      \item {This course covered the basics of machine learning using python, giving an introduction into topics like data exploration, model building, model validation, under and overfitting and random forests.}
      \item {Certificate of accomplishment: https://tinyurl.com/3t9xuxdy}
    \end{cvitems}
  }
  \newline

%---------------------------------------------------------
  \cventry
  {Kaggle.com, Online Machine-Learning Community} % Affiliation/role
  {\textit{Pandas}}
  {} % Location
  {2022} % Date(s)
  {
    \begin{cvitems} % Description(s) of experience/contributions/knowledge
      \item {Focused on enhancing data manipulation skills through coursework on the Pandas Python data analysis library.}
      \item {Certificate of accomplishment: https://tinyurl.com/4ec7n95c}
    \end{cvitems}
  }
  \newline
  %---------------------------------------------------------
  \cventry
  {Kaggle.com, Online Machine-Learning Community} % Affiliation/role
  {\textit{Feature Engineering}}
  {} % Location
  {2022} % Date(s)
  {
    \begin{cvitems} % Description(s) of experience/contributions/knowledge
      %\item {Not understanding a thing during my brother's phd defense about support vector machines inspired me to take this course.}
      \item {Learned techniques in feature engineering for machine learning tasks, including mutual information-based feature selection, feature creation, clustering, PCA, and target encoding for categorical data.}
      \item {Certificate of accomplishment: https://tinyurl.com/2u7kbxyh}
    \end{cvitems}
  }
  \newline

%---------------------------------------------------------
\cventry
{Kaggle.com, Online Machine-Learning Community} % Affiliation/role
{\textit{Data Visualization}}
{} % Location
{2022} % Date(s)
{
  \begin{cvitems} % Description(s) of experience/contributions/knowledge
    %\item {Not understanding a thing during my brother's phd defense about support vector machines inspired me to take this course.}
    \item {Applied python seaborn library for statistical data visualization, enhancing my data presentation skills.}
    \item {Certificate of accomplishment: https://tinyurl.com/bdee5xwk}
  \end{cvitems}
}
\newline


%---------------------------------------------------------
\cventry
  {John Hopkins University, Online} % Affiliation/role
  {\textit{Computing for Data Analysis}} % Organization/group
  {Grade: 97\%} % Date(s)
  {} % Location
  {
    \begin{cvitems} % Description(s) of experience/contributions/knowledge
      \item {Studied data analysis and visualization techniques in R within the context of medical data.}
      \item {Certificate of accomplishment: http://tinyurl.com/2ez3unkj}
    \end{cvitems}
  }
  \newline
%---------------------------------------------------------

  \cventry
  {Seqera.io, Online} % Affiliation/role
  {\textit{Nextflow and nf-core community training}} % Organization/group
  {} % Date(s)
  {2023} % Location
  {
    \begin{cvitems} % Description(s) of experience/contributions/knowledge
      \item {Participated in a training that introduced the scientific workflow system Nextflow and nf-core community pipelines.}
      \item {Learning about basic deployment and development of Nextflow nf-core pipelines.}
    \end{cvitems}
  }
  \newline


%---------------------------------------------------------
\cventry
  {Udemy, Online} % Affiliation/role
  {\textit{The Complete JavaScript Course 2023: From Zero to Expert!}} % Organization/group
  {} % Date(s)
  {2023} % Location
  {
    \begin{cvitems} % Description(s) of experience/contributions/knowledge
      \item {Completed a comprehensive 69-hour course in JavaScript development, gaining proficiency in both fundamental and advanced concepts.}
      \item {Actively applied Vanilla JavaScript to create small web applications using the Model View Controller (MVC) design pattern.}
      \item {Curriculum: https://www.udemy.com/course/the-complete-javascript-course/}
      \item {Certificate of accomplishment: https://tinyurl.com/mrxc5h3x}
    \end{cvitems}
  }
  \newline

%---------------------------------------------------------
\cventry
  {Udemy, Online} % Affiliation/role
  {\textit{Node.js, Express, MongoDB \& More}} % Organization/group
  {} % Date(s)
  {2023} % Location
  {
    \begin{cvitems} % Description(s) of experience/contributions/knowledge
      \item {This course covers backend engineering in JavaScript and focuses on building a secure and versatile RESTful API using Node.js, Express.js, MongoDB and more.}
      \item {Curriculum: https://www.udemy.com/course/nodejs-express-mongodb-bootcamp/}
    \end{cvitems}
  }
  \newline


  
%---------------------------------------------------------
% \cventry
%   {Self-taught} % Affiliation/role
%   {\textit{Full stack web development}} % Organization/group
%   {} % Date(s)
%   {} % Location
%   {
%     \begin{cvitems} % Description(s) of experience/contributions/knowledge
%       \item {Proficient in setting up and administering headless Linux systems}
%       \item {Applied SQL and NoSQL databases (e.g., Elasticsearch) in web development.}
%       \item {UUtilized Python Model-View-Controller (MVC) frameworks, including Django and Flask with SQLAlchemy.}
%       \item {Acquired skills in HTML, CSS, templating, and JavaScript for front-end development.}
%     \end{cvitems}
%   }
%---------------------------------------------------------
\end{cventries}



%-------------------------------------------------------------------------------
\end{document}
